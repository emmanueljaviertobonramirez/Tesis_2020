\chapter{Conclusiones}
	
	En este trabajo de tesis se analizaron tres topologías para diseñar integradores de orden fraccionario. La primera consistió en utilizar un filtro bilineal el cual al ser configurado en su forma polo y cero permitió implementar integradores con ordenes dentro del rango $[0.1, 0.81]$, ordenes superiores  no fueron posibles de implementar debido a las restricciones que presenta el hardware en cuanto a la frecuencia de reloj.
	
	La segunda consistió en combinar dos filtros, un pasabajas y un pasaaltas en una configuración de suma, esta topología permitió llegar a rangos de $[0.01, 0.93]$, no obstante presentando un poco más de error en la implementación con respecto al polo y cero. Esta topología es la más versátil debido a que el rango es mayor y su metodología de diseño es sencilla.
	
	La topología bicuadrática aunque teniendo menor error, su rango es muy pequeño bajo el esquema propuesto, de $[0.01, 0.6]$, esto lo hace poco útil en la mayoría de aplicaciones. Sin embargo es una topología que aun puede estudiarse y desarrollar diversos esquemas que pueden mejorar su rendimiento.
	
	Topologías de orden superior no son recomendables debido a que la mejora de poseen no es sustancial y seria un desperdicio de recursos.
	
	Finalmente la implementación del oscilador caótico resultó exitosa y se comprobó que es posible utilizar hardware analógico embebido para implementar sistemas de orden fraccionario además de generar reglas de diseño para futuros trabajos.