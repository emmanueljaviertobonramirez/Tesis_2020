\chapter{Resumen}


	En este trabajo de tesis se presenta en primera instancia el análisis teórico de la expansión de fracciones continuas (CFE) para su utilización en la creación de integradores de orden fraccionario. Se hace un análisis del error que esta aproximación presenta y se generará una metodología de diseño de acuerdo al orden fraccionario.
	
	Utilizando hardware analógico embebido (FPAA) y por medio del dispositivo NI ELVIS II+ se realizará la implementación física de los integradores de orden fraccionario y se medirá su respuesta en frecuencia. Obtenidos los resultados se obtendrá un análisis comparativo teórico contra experimental y se desarrollaran gráficas de mérito para facilitar el diseño.
	
	Finalmente se utilizarán las metodologias obtenidas para realizar la implementación de un oscilador caótico de orden fraccionario poniendo a prueba así el rendimiento de los integradores.