\chapter{Fundamentos teóricos}		
	Cuando se comienza a estudiar cálculo de orden entero es necesario familiarizarse con la notación de los operadores matemáticos de la derivada y la integral, con el cálculo fraccionario ocurre lo mismo. En la actualidad la notación más utilizada para el cálculo entero es la dada por Leibniz en (1686), donde el operador diferencial de n-ésimo orden esta definido como: $\frac{d^{n}}{dt^{n}}$, $D_{t}^{n}$ o simplemente $D^{n}$ con $n \in \mathbb{N}$. Utilizando el mismo razonamiento, puede definirse su operador inverso (antiderivada) de manera que el operador inverso de la derivada de n-ésimo orden está dado por: $_{a}D^{-n}_{t}$, donde $n \in \mathbb{N}$ y $a \in \mathbb{R}$ representa el límite inferior del dominio de la región donde se aplica dicho operador.
			
	Para generalizar el operador diferencial e integral para orden fraccionario se considera que este puede definirse para parámetros de orden real o incluso complejo. Esto implica que los operadores pueden definirse respectivamente como: $D^{\alpha}$ y $_{a}D^{\alpha}_{t}$ con $ \alpha \in \mathbb{R}$. 
		
	Es importante tener presente que no una hay una única definición para los operadores diferencial e integral fraccional, sino varias expresiones definidas por diferentes autores, entre las mas usadas se encuentran la definición de Grünwald-Letnikov (GL), la de Riemann-Liouville (RL) y la de Caputo (Ca), cada una de estas con sus ventajas y desventajas desde el punto de vista del análisis matemático, complejidad computacional e implementación \cite{Petras2011}.
			
	\section{Definición de Grünwald-Letnikov}
	
		\subsection{Definición de derivada de Grünwald-Letnikov}
			
		\subsection{Definición de integral de Grünwald-Letnikov}

		\subsection{Método numérico para la definición de GL}

	\section{Definición de Riemann-Liouville}

		\subsection{Definición de integral de Riemann-Liouville}
		
		\subsection{Definición de derivada de Riemann-Liouville}
		
	\section{Transformada de Laplace de integrales y derivadas fraccionarias}
	
	\section{Expansión de fracciones continuas (CFE)}\label{sec:CFE}
	
	\subsection{Análisis de error de la CFE}
	                                                                 
	\section{Escalamiento en frecuencia}

	\section{Teoría de filtros}
	
		\subsection{Filtros de primer orden}
	
		\subsection{Filtros de segundo orden}
	